% Intended LaTeX compiler: pdflatex
\documentclass[a4paper,12pt]{article}
\usepackage[utf8]{inputenc}
\usepackage[T1]{fontenc}
\usepackage{graphicx}
\usepackage{grffile}
\usepackage{longtable}
\usepackage{wrapfig}
\usepackage{rotating}
\usepackage[normalem]{ulem}
\usepackage{amsmath}
\usepackage{textcomp}
\usepackage{amssymb}
\usepackage{capt-of}
\usepackage{hyperref}
\usepackage{booktabs}
\usepackage{indentfirst}
\usepackage{graphicx}
\usepackage{todonotes}
\usepackage{hyperref}
\usepackage[utf8]{inputenc}
\usepackage{xcolor}
\definecolor{bg}{rgb}{0.98,0.98,0.98}
\usepackage{minted}
\setminted{
mathescape,
linenos,
numbersep=5pt,
frame=lines,
framesep=2mm,
autogobble,
style=tango,
bgcolor=bg
}
\usepackage{sectsty}
\allsectionsfont{\sffamily}
\usepackage{amssymb}
\usepackage[normalem]{ulem}
\renewcommand{\d}{\ensuremath{\mathrm{d}}}
\setlength{\textheight}{21cm}
\setlength{\textwidth}{16cm}
\setlength{\evensidemargin}{-0cm}
\setlength{\oddsidemargin}{-0cm}
\setlength{\topmargin}{0cm}
%\renewcommand{\baselinestretch}{1.2}
%\renewcommand{\topfraction}{0.8}
%\renewcommand{\bottomfraction}{0.6}
%\renewcommand{\textfraction}{0.2}
\usepackage[round]{natbib}
\usepackage{xeCJK}
\usepackage[round]{natbib}
\setCJKmainfont{Noto Sans CJK SC}
\setCJKsansfont{Noto Sans CJK SC}
\setCJKmonofont{Noto Sans Mono CJK SC}
% Generate PDF: xelatex -shell-escape <tex file>
\author{jouyouyun}
\date{\today}
\title{ABI 结构体变动测试}
\hypersetup{
 pdfauthor={jouyouyun},
 pdftitle={ABI 结构体变动测试},
 pdfkeywords={},
 pdfsubject={},
 pdfcreator={Emacs 28.0.50 (Org mode 9.3)},
 pdflang={English}}
\begin{document}

\maketitle
\newpage

\tableofcontents
\newpage


一个应用 \texttt{AppA} 使用了动态库 \texttt{LibA} ,此时更新 \texttt{LibA} 的结构体成员,在中间位置添加成员,需要重新编译生成 \texttt{AppA} 才能得到正确的结果。\\

本文针对这个问题进行了验证,具体代码见下文。

\newpage

\section{代码}
\label{sec:orgdbc04b8}

\subsection{设计说明}
\label{sec:org6304594}

设计了一个动态库 \texttt{libtest} ,里面包含了一个结构体,结构体里有 \texttt{2} 个 \texttt{int} 成员,另外提供了一个创建此结构体的方法,创建结构体对每个成员指定了默认值。\\

复制了上述的动态库,在结构体成员之间添加了一个新的成员,创建接口里照样指定了新成员的默认值。\\

编写了一个应用,基于原始动态库进行编译,应用执行后会打印原始动态库成员的值。使用 \texttt{LD\_LIBRARY\_PATH} 指定不同的 \texttt{libtest} 路径来运行应用,输出的结果应该不一致。
证明结构体成员变动后需要重新编译应用。\\

\subsection{目录结构}
\label{sec:org8e5d34e}

\begin{minted}{shell}
./
├── changed
│   ├── Makefile
│   ├── test.c
│   └── test.h
├── main.c
├── Makefile
├── README.org
└── origin
    ├── Makefile
    ├── test.c
    └── test.h
\end{minted}

\subsection{文件说明}
\label{sec:orgabdba85}

\begin{itemize}
\item \texttt{origin}

原始的动态库的代码

\item \texttt{changed}

添加了成员的动态库代码

\item \texttt{main.c}

测试程序代码
\end{itemize}

\subsection{测试步骤}
\label{sec:org090bef6}

执行以下命令进行测试:

\begin{minted}{shell}
make build
make run-origin
make run-changed
\end{minted}

输出结果如下:
\begin{minted}{shell}
$ make build
cd ./origin && make rebuild && cd ..
make[1]: 进入目录“./origin”
rm -f libtest.so
gcc -shared -fPIC -o libtest.so test.c
make[1]: 离开目录“./origin”
cd ./changed && make rebuild && cd ..
make[1]: 进入目录“./changed”
rm -f libtest.so
gcc -shared -fPIC -o libtest.so test.c
make[1]: 离开目录“./changed”
gcc -Wall -g -L./origin -ltest -o abi_test main.c

$ make run-origin
env LD_LIBRARY_PATH=./origin ./abi_test
Test: (1, 3)

$ make run-changed
env LD_LIBRARY_PATH=./changed ./abi_test
Test: (1, 2)
\end{minted}

\newpage

\subsection{代码内容}
\label{sec:orgbc96ad9}

\subsubsection{origin}
\label{sec:org923fa06}

\begin{itemize}
\item \texttt{test.h}
\end{itemize}

\begin{minted}{c}
#ifndef __TEST_API_H__
#define __TEST_API_H__

struct test_p {
    int a;
    int b;
};

struct test_p *create_test(void);

#endif
\end{minted}


\begin{itemize}
\item \texttt{test.c}
\end{itemize}

\begin{minted}{c}
#include "test.h"

#include <stdlib.h>

struct test_p*
create_test(void)
{
    struct test_p *p = calloc(1, sizeof(struct test_p));
    if (!p)
        return NULL;

    p->a = 1;
    p->b = 3;

    return p;
}
\end{minted}

\begin{itemize}
\item \texttt{Makefile}
\end{itemize}

\begin{minted}{makefile}
CC = gcc
TARGET = libtest.so
SRC := test.c

build:
    ${CC} -shared -fPIC -o ${TARGET} ${SRC}

clean:
    rm -f ${TARGET}

rebuild: clean build
\end{minted}

\newpage

\subsubsection{changed}
\label{sec:orgb3afe66}

这里只给出相对 \texttt{origin} 变动的代码:

\begin{itemize}
\item \texttt{test.h}
\end{itemize}

\begin{minted}{diff}
$ diff origin/test.h changed/test.h
5a6
>     int aa;
\end{minted}

\begin{itemize}
\item \texttt{test.c}
\end{itemize}

\begin{minted}{diff}
$ diff origin/test.c changed/test.c
12a13
>     p->aa = 2;
\end{minted}

\newpage

\subsubsection{app}
\label{sec:org08c21e4}

\begin{itemize}
\item \texttt{main.c}
\end{itemize}

\begin{minted}{c}
#include <stdio.h>
#include <stdlib.h>

#include "./origin/test.h"

int
main(int argc, char *argv[])
{
    if (argc > 1) {
        printf("Usage: %s", argv[0]);
        return -1;
    }

    struct test_p *p = create_test();
    if (!p) {
        printf("Failed to create test\n");
        return -2;
    }

    printf("Test: (%d, %d)\n", p->a, p->b);
    free(p);

    return 0;
}
\end{minted}

\begin{itemize}
\item \texttt{Makefile}
\end{itemize}

\begin{minted}{makefile}
CC = gcc
TARGET = abi_test
ORIGIN = ./origin
CHANGED = ./changed

build :
    cd ${ORIGIN} && make rebuild && cd ..
    cd ${CHANGED} && make rebuild && cd ..
    gcc -Wall -g -L${ORIGIN} -ltest -o ${TARGET} main.c

run-origin:
    env LD_LIBRARY_PATH=${ORIGIN} ./${TARGET}

run-changed:
    env LD_LIBRARY_PATH=${CHANGED} ./${TARGET}

clean:
    cd ${ORIGIN} && make clean && cd ..
    cd ${CHANGED} && make clean && cd ..
    rm -f ${TARGET}

rebuild: clean build
\end{minted}
\end{document}
